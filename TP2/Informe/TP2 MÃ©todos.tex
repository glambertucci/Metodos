\documentclass[a4paper]{article}
\usepackage[utf8]{inputenc}
\usepackage[spanish, es-tabla]{babel}

\usepackage{amsmath}
\usepackage{amsfonts}
\usepackage{amssymb}

\usepackage{float}
\usepackage{graphicx}
\graphicspath{ {./Imagenes/} }

\usepackage{fancyhdr}

\usepackage{units} 

\pagestyle{fancy}
\fancyhf{}
\lhead{93.54 Métodos Numéricos}
\rhead{Fontecha, Lambertucci, Pouthier, Londero B.}
\rfoot{Página \thepage}



\begin{document}

%%%%%%%%%%%%%%%%%%%%%%%%%%%%%%%%%%%%%%%%%%%%%%%%%%%%%%%%%%%%%%%%%%%%%%%%% 
%								CARATULA								%
%%%%%%%%%%%%%%%%%%%%%%%%%%%%%%%%%%%%%%%%%%%%%%%%%%%%%%%%%%%%%%%%%%%%%%%%% 

\begin{titlepage}
\newcommand{\HRule}{\rule{\linewidth}{0.5mm}}
\center
\mbox{\textsc{\LARGE \bfseries {Instituto Tecnológico de Buenos Aires}}}\\[1.5cm]
\textsc{\Large 93.54 Métodos Numéricos}\\[0.5cm]


\HRule \\[0.6cm]
{ \Huge \bfseries Trabajo práctico N$^{\circ}$2}\\[0.4cm] 
\HRule \\[1.5cm]


{\large

\emph{Grupo 3}\\
\vspace{3px}

\begin{tabular}{lr} 	
\textsc{Fontecha}, María Eugenia  & 58138 \\
\textsc{Lambertucci}, Guido Enrique  & 58009 \\
\textsc{Pouthier}, Florian  & 61337 \\
\textsc{Londero Bonaparte}, Tomás Guillermo  & 58150 \\
\end{tabular}

\vspace{20px}

\emph{Profesor}\\
\vspace{3px}
\textsc{Fierens}, Pablo Ignacio\\ 	
\textsc{Álvarez}, Adrián Omar\\ 

\vspace{100px}

\begin{tabular}{ll}

Presentado: & 02/06/19\\

\end{tabular}

}

\vfill

\end{titlepage}


%%%%%%%%%%%%%%%%%%%%%%%%%%%%%%%%%%%%%%%%%%%%%%%%%%%%%%%%%%%%%%%%%%%%%%%%% 
%								INFORME									%
%%%%%%%%%%%%%%%%%%%%%%%%%%%%%%%%%%%%%%%%%%%%%%%%%%%%%%%%%%%%%%%%%%%%%%%%%


\section{Introducción}

El trabajo presentado consiste en aproximar funciones dadas en ecuaciones diferenciales, mediante el uso del método de Heun o tambien conocido como Runge-Kutta de segundo orden. Estas ecuaciones representan un modelo de crecimiento óseo basado en un balance entre osteoblastos y osteoclastos, obtenidas delel artículo de \textit{Lemaire et al}\footnote{Vincent Lemaire, Frank L. Tobin, Larry D. Greller, Carolyn R. Cho, and
Larry J. Suva. Modeling the interactions between osteoblast and osteoclast
activities in bone remodeling. 229:293–309, 2004.}.

El sistema a aproximar es el siguiente:

\begin{equation}
\frac{dR}{dt} \ = \ D_R \cdot {\pi}_{C} \ - \ \frac{D_B}{{\pi}_{C}} \ \ {I}_{R},
\end{equation}

\begin{equation}
\frac{dB}{dt} \ = \ \frac{D_B}{{\pi}_{C}} \cdot R \ - \ k_B \cdot B \ \ + {I}_{B},
\end{equation}

\begin{equation}
\frac{dC}{dt} \ = \ D_C \cdot {\pi}_{L} \ - \ D_A \cdot	{\pi}_{C} \cdot C \ \ + \ \ {I}_{C}.
\end{equation}

donde ${D}_{A}$, ${D}_{B}$, ${D}_{C}$ y ${D}_{R}$ son constantes, $R$, $B$ y $C$ las incógnitas ${\pi}_{C}$ y ${\pi}_{L}$ son funciones de las incógnitas (${\pi}_{C} $ función de $ C $ y $ {\pi}_{L} $ función de $B$ y $R$). A su vez, estas ultimas dependen de otras funciones, las cuales son dato y varían dependiendo la situación analizada. Todos los valores, tanto de las constantes como de las funciones anteriormente mencionadas, son obtenidas del artículo.

\section{Código empleado}
El código elaborado para este trabajo fue hecho en \textbf{Matlab}.
Cabe destacar que se probó previamente el algoritmo aproximando laa ecuación diferencial

\[\left\{
  \begin{array}{lr}
    \frac{dx}{dt} \ = \ x, \ \ \ \ \ \ \ con \ t \in [0;1]\\
    x(0) = 1
  \end{array}
\right.
\]

sabiendo que el resultado de esta es

\[ x(t) \ = \ e^t \]

determinando así su adecuado funcionamiento.
	
	
	
\section{Resultados obtenidos}
A continuación, se presentan los resultados obtenidos de las funciones estimadas. En estas se observa la concentración de células (en picomoles) en función del tiempo (en días). 

%\begin{figure}[H]
%	\centering
%	\includegraphics[width=0.9\textwidth]{.png}
%	\caption{.}
%\end{figure}	


\section{Conclusión}
Comparando los gráficos obtenidos con los presentados en el artículo, se puede observar que las aproximaciones, realizadas con una tolerancia de $10^{-6}$, son buenas, es decir, próximas a las empleadas en el artículo. 	
 
\end{document}
